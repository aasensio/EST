% \documentclass[a4paper,10pt]{report}
% \usepackage[utf8x]{inputenc}
% \usepackage[left=1.0in, right=1.0in, top=1.0in, bottom=1.0in]{geometry}
% \usepackage{tikz}
\documentclass[iop,onecolumn]{emulateapj}

\usepackage{amssymb}
\usepackage{amsmath}
\usepackage{natbib}
\usepackage{algorithm2e}
\newcommand{\mubold}{\mbox{\boldmath$\mu$}}
\newcommand{\omegabold}{\mbox{\boldmath$\omega$}}
\newcommand{\alphabold}{\mbox{\boldmath$\alpha$}}
\newcommand{\betabold}{\mbox{\boldmath$\beta$}}
\newcommand{\phibold}{\mbox{\boldmath$\phi$}}
\newcommand{\epsilonbold}{\mbox{\boldmath$\epsilon$}}
\newcommand{\thetabold}{\mbox{\boldmath$\theta$}}
\newcommand{\gammabold}{\mbox{\boldmath$\gamma$}}
\newcommand{\psibold}{\mbox{\boldmath$\psi$}}
\newcommand{\sigmabold}{\mbox{\boldmath$\sigma$}}
\newcommand{\argmax}[1]{\underset{#1}{\operatorname{arg}\,\operatorname{max}}\;}
\DeclareMathOperator*{\argmin}{arg\,min}
\DeclareMathOperator*{\prox}{\mathrm{prox}}

\begin{document}
% Title Page
\title{Correlating Shack-Hartmann}
\author{A. Asensio Ramos}

\maketitle

%%%%%%%%%%%%%%%%%%%%%%%%%%%%%%%%%%%%%%%%%%%%%%%%%%%%%%%%%%%%%%%%%%%%%%%%%%%%%%%%%%%%%%%%%%%%%
%%%%%%%%%%%%%%%%%%%%%%%%%%%%%%%%%%%%%%%%%%%%%%%%%%%%%%%%%%%%%%%%%%%%%%%%%%%%%%%%%%%%%%%%%%%%%
\section{Analytical method}
%%%%%%%%%%%%%%%%%%%%%%%%%%%%%%%%%%%%%%%%%%%%%%%%%%%%%%%%%%%%%%%%%%%%%%%%%%%%%%%%%%%%%%%%%%%%%
%%%%%%%%%%%%%%%%%%%%%%%%%%%%%%%%%%%%%%%%%%%%%%%%%%%%%%%%%%%%%%%%%%%%%%%%%%%%%%%%%%%%%%%%%%%%%
%%%%%%%%%%%%%%%%%%%%%%%%%%%%%%%%%%%%%%%%%%%%%%%%%%%%%%%%%%%%%%%%%%%%%%%%%%%%%%%%%%%%%%%%%%%%%
Given two images $f(\mathbf{r})$ and $g(\mathbf{r}+\mathbf{r}_0)$, with $\mathbf{r}=(x,y)$
and $\mathbf{r}_0$, we can estimate the shift between the two images assuming that they
are similar by optimizing the following squared error:
\begin{equation}
\epsilon^2 = \int \int_{-\infty}^{\infty} \left| f(\mathbf{r}) - g(\mathbf{r}+\mathbf{r}_0)\right|^2 \mathrm{d}\mathbf{f}= 
\int \int_{-\infty}^{\infty} \left| F(\mathbf{f}) - G(\mathbf{f}) \exp \left( 2\pi i \,\mathbf{f} \cdot \mathbf{r}
\right) \right|^2 \mathrm{d}\mathbf{f},
\end{equation}
where the last equation is a consequence of the Parseval's theorem and capital
letters refer to the Fourier transform of the equivalent functions in small letters. If we want to estimate
the shift of many images with respect to a reference one, one can build a merit function
by adding all squared errors:
\begin{equation}
\epsilon^2 = \sum_{i=1}^N  
\int \int_{-\infty}^{\infty} \left| F_i(\mathbf{f}) - G_i(\mathbf{f}) \exp \left( 2\pi i \,\mathbf{f} \cdot \mathbf{r}_i
\right) \right|^2 \mathrm{d}\mathbf{f}.
\label{eq:merit}
\end{equation}
Given that every individual $\mathbf{r}_i$ affects only the $i$-th term in $\epsilon^2$, solving
for all subapertures simultaneously in this way is equivalent to finding each $\mathbf{r}_i$
that optimizes each $\epsilon^2$.

However, one can introduce some contraints by assuming that the displacement in all
subapertures are not independent, but come from a common wavefront. We assume that the wavefront
$\phi(x,y)$ at any point in the pupil is assumed to be given as a linear combination of 
Zernike polynomials, so that:
\begin{equation}
\phi(x,y) = \sum_{j=1}^M \alpha_j Z_j(x,y),
\end{equation}
where $j$ is the Noll index and $Z_j(x,y)$ is the Zernike polynomial. A Shack-Hartmann (SH) wavefront
sensor is a device that consists of several subapertures that produce images of the same
object affected by different parts of the wavefront. The image is then shifted to a position
$(\Delta x_i,\Delta y_i)$ that is given by the horizontal and vertical derivative of the wavefront:
\begin{align}
\Delta x_i = F \frac{\partial \phi(x_i,y_i)}{\partial x} = \sum_{j=1}^M \alpha_j \frac{\partial Z_j(x_i,y_i)}{\partial x}, \nonumber \\
\Delta y_i = F \frac{\partial \phi(x_i,y_i)}{\partial y} = \sum_{j=1}^M \alpha_j \frac{\partial Z_j(x_i,y_i)}{\partial y},
\end{align}
where $F$ is the focal of the SH microlenses.

The previous expressions can then be plugged into Eq. (\ref{eq:merit}) to find:
\begin{equation}
\epsilon^2 = \sum_{i=1}^N  
\int \int_{-\infty}^{\infty} \left| F_i(\mathbf{f}) - G_i(\mathbf{f}) \exp \left[ 2\pi i \sum_j \alpha_j \left(
f_x Z_{ij}^x + f_y Z_{ij}^y \right) \right] \right|^2 \mathrm{d}\mathbf{f},
\end{equation}
where $Z_{ij}^x$ is the horizontal derivative of the $j$ Zernike polynomial for subaperture $i$.
For simplifying the notation during the calculations, we make the substitution $K_{ij}=f_x Z_{ij}^x + f_y Z_{ij}^y$.
The previous merit function can then be optimized by taking the derivative with respect to all
$\alpha_j$ and equate to zero. It can be checked that each derivative is given by:
\begin{equation}
\frac{\partial \epsilon^2}{\partial \alpha_k} = \Re \left\{ 4\pi i   
\int \int_{-\infty}^{\infty} \sum_{i=1}^N \left[ K_{ik} F_i G_i^* \exp \left( -2\pi i \sum_j \alpha_j 
K_{ij} \right) \right] \mathrm{d}\mathbf{f} \right\}=0.
\end{equation}
Now, the product $F_i G_i^*$ is the cross-correlation of the two functions. The cross-correlation
function can be written, in general, as 
\begin{equation}
C_i(\mathbf{f}) = F_i G_i^* = |C(\mathbf{f})| \exp \left[ i \phi_i(\mathbf{f}) \right].
\end{equation}
This allows us to rewrite each equation as:
\begin{equation}
\frac{\partial \epsilon^2}{\partial \alpha_k} = \Re \left\{ i   
\int \int_{-\infty}^{\infty} \sum_{i=1}^N \left[ K_{ik} |C_i| \exp \left( -2\pi i \sum_j \alpha_j 
K_{ij} + i \phi_i \right) \right] \mathrm{d}\mathbf{f} \right\}=0.
\end{equation}
It is obvious that, if the phase of the cross-correlation is obtained with precision, it has
to be very similar to the sum over the Zernike coefficients. This allows us to make a Taylor
expansion and cut at first order, so that we approximate:
\begin{equation}
\exp \left( -2\pi i \sum_j \alpha_j 
K_{ij} + \phi_i \right) \approx 1 + i \phi_i -2\pi i \sum_j \alpha_j K_{ij}.
\end{equation}
This allows us to linearize the equations and transform them into a linear system of
equations. Plugging the series expansion into the derivative expression, we find:
\begin{equation}
\frac{\partial \epsilon^2}{\partial \alpha_k} = \Re \left[  i   
\int \int_{-\infty}^{\infty} \sum_{i=1}^N K_{ik} |C_i| \mathrm{d}\mathbf{f} \right]
- \Re \left[   
\int \int_{-\infty}^{\infty} \sum_{i=1}^N K_{ik} |C_i| \phi_i \mathrm{d}\mathbf{f} \right]
+ \Re \left[ 2\pi    
\int \int_{-\infty}^{\infty} \sum_{i=1}^N K_{ik} |C_i| \sum_j \alpha_j 
K_{ij} \mathrm{d}\mathbf{f} \right].
\end{equation}

The first term is obviously zero because the integrand is real. The second and third terms can
be worked out and written as:
\begin{equation}
-\int \int_{-\infty}^{\infty} \sum_{i=1}^N K_{ik} |C_i| \phi_i \mathrm{d}\mathbf{f}
= -\sum_{i=1}^N \left( Z_{ik}^x b_i^x + Z_{ik}^y b_i^y \right) = -b_k,
\end{equation}
and
\begin{equation}
2\pi \int \int_{-\infty}^{\infty} \sum_{i=1}^N K_{ik} |C_i| \sum_j \alpha_j K_{ij} \mathrm{d}\mathbf{f}=
2\pi \sum_j A_{kj} \alpha_j,
\end{equation}
where
\begin{equation}
A_{kj} = \sum_i Z_{ik}^x Z_{ij}^x a^{xx}_i + \sum_i \left( Z_{ik}^x Z_{ij}^y + Z_{ik}^y Z_{ij}^x \right) a^{xy}_i + 
\sum_i Z_{ik}^y Z_{ij}^y a^{yy}_i,
\end{equation}
and
\begin{align}
a^{xx}_i &= \int \int_{-\infty}^{\infty} |C_i| f_x^2 \mathrm{d}\mathbf{f} \\
a^{yy}_i &= \int \int_{-\infty}^{\infty} |C_i| f_y^2 \mathrm{d}\mathbf{f} \\
a^{xy}_i &= \int \int_{-\infty}^{\infty} |C_i| f_x f_y \mathrm{d}\mathbf{f} \\
b^{x}_i &= \int \int_{-\infty}^{\infty} |C_i| \phi_i f_x \mathrm{d}\mathbf{f} \\
b^{y}_i &= \int \int_{-\infty}^{\infty} |C_i| \phi_i f_y \mathrm{d}\mathbf{f}
\end{align}

Summarizing, the coefficients of the wavefront are obtained by solving the 
linear system:
\begin{equation}
\sum_j A_{kj} \alpha_j = \frac{-b_k}{2\pi},
\end{equation}
which can be written in matrix form as:
\begin{equation}
\mathbf{A} \alphabold = \mathbf{b}.
\end{equation}

\bibliographystyle{aa}
\bibliography{/scratch/Dropbox/biblio}


\end{document}          
