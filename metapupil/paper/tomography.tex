\documentclass[]{aa} % for the letters 
\usepackage{txfonts,epsfig,graphicx,natbib,url,twoopt,amsmath,algorithm2e}
\usepackage{color}

\newcommand{\vectornorm}[1]{\left|\left|#1\right|\right|}
\newcommand{\mubold}{\mbox{\boldmath$\mu$}}
\newcommand{\thetabold}{\mbox{\boldmath$\theta$}}
\newcommand{\betabold}{\mbox{\boldmath$\beta$}}
\newcommand{\etabold}{\mbox{\boldmath$\eta$}}
\newcommand{\epsilonbold}{\mbox{\boldmath$\epsilon$}}
\newcommand{\lambdabold}{\mbox{\boldmath$\lambda$}}
\newcommand{\Thetabold}{\mbox{\boldmath$\Theta$}}
\newcommand{\Sigmabold}{\mbox{\boldmath$\Sigma$}}
\newcommand{\sigmabold}{\mbox{\boldmath$\sigma$}}
\newcommand{\Lambdabold}{\mbox{\boldmath$\Lambda$}}
\newcommand{\Omegabold}{\mbox{\boldmath$\Omega$}}
\newcommand{\alphabold}{\mbox{\boldmath$\alpha$}}
\newcommand{\Phibold}{\mbox{\boldmath$\Phi$}}
\newcommand{\omegabold}{\mbox{\boldmath$\omega$}}
\newcommand{\phibold}{\mbox{\boldmath$\phi$}}
\newcommand{\hazel}{\textsc{Hazel}}
\DeclareMathOperator*{\argmin}{arg\,min}
\DeclareMathOperator*{\prox}{\mathrm{prox}}
\def\comment#1{\textcolor{red}{[#1]}}

%%%%%%%%%%%%%%%%%%%%%%%%%%%%%
%%% Results are in /scratch/Dropbox/THEORY/inversionWavelet/forPaper
%%%%%%%%%%%%%%%%%%%%%%%%%%%%%

\usepackage{natbib,twoopt}
% \usepackage[breaklinks=true]{hyperref} %% to avoid \citeads line fills
\bibpunct{(}{)}{;}{a}{}{,}             %% natbib format for A&A and ApJ
\makeatletter
  \newcommandtwoopt{\citeads}[3][][]{\href{http://adsabs.harvard.edu/abs/#3}%
    {\def\hyper@linkstart##1##2{}%
     \let\hyper@linkend\@empty\citealp[#1][#2]{#3}}}
  \newcommandtwoopt{\citepads}[3][][]{\href{http://adsabs.harvard.edu/abs/#3}%
    {\def\hyper@linkstart##1##2{}%
     \let\hyper@linkend\@empty\citep[#1][#2]{#3}}}
  \newcommandtwoopt{\citetads}[3][][]{\href{http://adsabs.harvard.edu/abs/#3}%
    {\def\hyper@linkstart##1##2{}%
     \let\hyper@linkend\@empty\citet[#1][#2]{#3}}}
  \newcommandtwoopt{\citeyearads}[3][][]%
    {\href{http://adsabs.harvard.edu/abs/#3}
    {\def\hyper@linkstart##1##2{}%
     \let\hyper@linkend\@empty\citeyear[#1][#2]{#3}}}
\makeatother

\begin{document}

\title{Regularized MCAO tomography}

\author{A. Asensio Ramos\inst{1,2}}

\institute{
 Instituto de Astrof\'{\i}sica de Canarias, 38205, La Laguna, Tenerife, Spain; \email{aasensio@iac.es}
\and
Departamento de Astrof\'{\i}sica, Universidad de La Laguna, E-38205 La Laguna, Tenerife, Spain
}
             
  \date{Received ---; accepted ---} 

  \abstract{.}

   \keywords{stars: magnetic fields, atmospheres --- line: profiles --- methods: data analysis}
   \authorrunning{Asensio Ramos et al.}
   \titlerunning{}
   \maketitle
%
%________________________________________________________________

\section{Introduction}
Fluctuations in the refraction index of the atmosphere affects the propagation of 
light and produces what astronomers know as \emph{seeing}. Under the

\begin{acknowledgements}
Financial support by the Spanish Ministry of Economy and Competitiveness 
through projects AYA2014-60476-P, AYA2014-60833-P and Consolider-Ingenio 2010 CSD2009-00038 
are gratefully acknowledged. 
JdlCR is supported by grants from the
Swedish Research Council (Vetenskapsr\aa det) and the Swedish National Space Board.
AAR also acknowledges financial support through the Ram\'on y Cajal fellowship. 
This research has made use of NASA's Astrophysics Data System Bibliographic Services.
\end{acknowledgements}

% \bibliographystyle{aa}
% \bibliography{/scratch/Dropbox/biblio}

\end{document}
